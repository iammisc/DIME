\documentclass{article}

\newcommand{\filepath}[1]{\texttt{#1}}
\newcommand{\command}[1]{\texttt{#1}}

\begin{document}
\title{DIME: A distributed in-memory database}
\author{Travis Athougies}
\date{December 31, 2012}
\maketitle

\section{The Block Servers}

The block servers are the worker bees of the DIME hive. They store the blocks that make up the
columns the table server keeps track of. In general, the table server manages all communication with
the block servers. The table server also checks if any block servers have gone down, and will
automatically reroute requests based on this information.

The main server loop for the block servers can be found in
\filepath{Database/DIME/DataServer.hs}. The servers respond to the commands foundin
\filepath{Database/DIME/DataServer/Command.hs}. These commands are sent to the servers by the table
server.

The block servers blindly store blocks. They know nothing about how the blocks fit into the
overarching picture. Each block has an ID number, which is unique within the column. Since each
column is identified by a unique column ID within a table, each block can be uniquely identified by
a triple $(T, C, B),$ where $T$ is the table ID, $C$ is the column ID, and $B$ is a block ID.

The commands for manipulating blocks at this low-level are discussed below.

\subsection{\command{NewBlock}}

\end{document}